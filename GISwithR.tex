% !TEX TS-program = pdflatex
% !TEX encoding = UTF-8 Unicode

% This is a simple template for a LaTeX document using the "article" class.
% See "book", "report", "letter" for other types of document.

\documentclass[11pt]{article} % use larger type; default would be 10pt

\usepackage[utf8]{inputenc} % set input encoding (not needed with XeLaTeX)

%%% Examples of Article customizations
% These packages are optional, depending whether you want the features they provide.
% See the LaTeX Companion or other references for full information.

%%% PAGE DIMENSIONS
\usepackage{geometry} % to change the page dimensions
\geometry{a4paper} % or letterpaper (US) or a5paper or....
% \geometry{margin=2in} % for example, change the margins to 2 inches all round
% \geometry{landscape} % set up the page for landscape
%   read geometry.pdf for detailed page layout information

\usepackage{graphicx} % support the \includegraphics command and options

% \usepackage[parfill]{parskip} % Activate to begin paragraphs with an empty line rather than an indent
\usepackage{framed}
%%% PACKAGES
\usepackage{booktabs} % for much better looking tables
\usepackage{array} % for better arrays (eg matrices) in maths
\usepackage{paralist} % very flexible & customisable lists (eg. enumerate/itemize, etc.)
\usepackage{verbatim} % adds environment for commenting out blocks of text & for better verbatim
\usepackage{subfig} % make it possible to include more than one captioned figure/table in a single float
% These packages are all incorporated in the memoir class to one degree or another...

%%% HEADERS & FOOTERS
\usepackage{fancyhdr} % This should be set AFTER setting up the page geometry
\pagestyle{fancy} % options: empty , plain , fancy
\renewcommand{\headrulewidth}{0pt} % customise the layout...
\lhead{}\chead{}\rhead{}
\lfoot{}\cfoot{\thepage}\rfoot{}

%%% SECTION TITLE APPEARANCE
\usepackage{sectsty}
\allsectionsfont{\sffamily\mdseries\upshape} % (See the fntguide.pdf for font help)
% (This matches ConTeXt defaults)

%%% ToC (table of contents) APPEARANCE
\usepackage[nottoc,notlof,notlot]{tocbibind} % Put the bibliography in the ToC
\usepackage[titles,subfigure]{tocloft} % Alter the style of the Table of Contents
\renewcommand{\cftsecfont}{\rmfamily\mdseries\upshape}
\renewcommand{\cftsecpagefont}{\rmfamily\mdseries\upshape} % No bold!

%%% END Article customizations

%%% The "real" document content comes below...

\title{Geographic Information Systems with \texttt{R}}
\author{Kevin O'Brien}

%http://journal.r-project.org/archive/2011-1/RJournal_2011-1_South.pdf
%http://cran.r-project.org/web/packages/gstat/vignettes/gstat.pdf


\begin{document}
\maketitle
\tableofcontents
\newpage
\section{Geographic Information Systems}

%What is GIS?


\section{Preliminaries}
\begin{itemize}
\item Working Directory
\item Packages
\end{itemize}

The package \textbf{\textit{sp}} provides general purpose classes and methods for defining, importing/exporting and visualizing spatial data.
\begin{framed}
\begin{verbatim}
setwd("~/UsingR-GIS")
 
### Basic packages ###
 
library(sp)             # classes for spatial data
library(raster)         # grids, rasters
library(rasterVis)      # raster visualisation
library(maptools)
			# and the dependencies
 \end{verbatim}
\end{framed}


\subsection{The Ozone Example}
% http://www.ats.ucla.edu/stat/r/faq/variogram.htm
Let's look at an example. Our dataset, ozone, contains ozone measurements from thirty-two locations in the Los Angeles area aggregated over one month. The dataset includes the station number (Station), the latitude and longitude of the station (Lat and Lon), and the average of the highest eight hour daily averages (Av8top). 

This data, and other spatial datasets, can be downloaded from the University of Illinois's Spatial Analysis Lab. By generating a variogram, we will be able to look at the variance of the differences of  Av8top among pairs of stations at different distances.  We can look at a sample of our data and then a summary of the distances between the stations.

\begin{framed}
\begin{verbatim}
ozone<-read.table("http://www.ats.ucla.edu/stat/r/faq/ozone.csv", sep=",", header=T)
head(ozone, n=10)
\end{verbatim}
\end{framed}
\newpage
\section{Visualisation}
\begin{verbatim} 
### VISUALISATION OF GEOGRAPHICAL DATA ###
 
 
### RWORLDMAP ###
 
library(rworldmap)   # visualising (global) spatial data
 
  # examples:
  newmap <- getMap(resolution="medium", projection="none")
  plot(newmap)
 
  mapCountryData()
  mapCountryData(mapRegion="europe")
  mapGriddedData()
  mapGriddedData(mapRegion="europe")
 
 
### GOOGLEVIS ###
 
library(googleVis)    # visualise data in a web browser using Google
Visualisation API
 
  # demo(googleVis)   # run this demo to see all the possibilities
 
  # Example: plot country-level data
  data(Exports)
  View(Exports)
  Geo <- gvisGeoMap(Exports, locationvar="Country", numvar="Profit",
                    options=list(height=400, dataMode='regions'))
  plot(Geo)
  print(Geo)
  # this HTML code can be embedded in a web page (and be dynamically updated!)
 
  # Example: Plotting point data onto a google map (internet)
  data(Andrew)
  M1 <- gvisMap(Andrew, "LatLong" , "Tip", options=list(showTip=TRUE,
showLine=F, enableScrollWheel=TRUE,
                           mapType='satellite', useMapTypeControl=TRUE,
width=800,height=400))
  plot(M1)
\end{verbatim}
 
%-----------------------------------------------------------%
\newpage
\subsection{GoogleMaps}

\texttt{Rgooglemaps}

\begin{verbatim} 
### RGOOGLEMAPS ###
 
library(RgoogleMaps)
 
  # get maps from Google
  newmap <- GetMap(center=c(36.7,-5.9), zoom =10, destfile = "newmap.png",
maptype = "satellite")
  # View file in your wd
  # now using bounding box instead of center coordinates:
  newmap2 <- GetMap.bbox(lonR=c(-5, -6), latR=c(36, 37), destfile =
"newmap2.png", maptype="terrain")    # try different maptypes
  newmap3 <- GetMap.bbox(lonR=c(-5, -6), latR=c(36, 37), destfile =
"newmap3.png", maptype="satellite")
 
  # and plot data onto these maps, e.g. these 3 points
   PlotOnStaticMap(lat = c(36.3, 35.8, 36.4), lon = c(-5.5, -5.6, -5.8), zoom=
10, cex=2, pch= 19, col="red", FUN = points, add=F)
 
 \end{verbatim}
\subsection{The \texttt{dismo} package}
\begin{verbatim}
### GMAP (DISMO) ###
 
library(dismo)
 
  # Some examples
  # Getting maps for countries
  mymap <- gmap("France")   # choose whatever country
  plot(mymap)
  mymap <- gmap("Spain", type="satellite")   # choose map type
  plot(mymap)
  mymap <- gmap("Spain", type="satellite", exp=3)  # choose the zoom level
  plot(mymap)
  mymap <- gmap("Spain", type="satellite", exp=8)
  plot(mymap)
 
  mymap <- gmap("Spain", type="satellite", filename="Spain.gmap")    # save the
map as a file in your wd for future use
 
  # Now get a map for a region drawn at hand
  mymap <- gmap("Europe")
  plot(mymap)
  select.area <- drawExtent()   # now click on the map to select your region
  mymap <- gmap(select.area)
  plot(mymap)
  # See ?gmap for many other possibilities
 
\end{verbatim}
\newpage


\newpage
\section{Interacting with other GIS}
\begin{framed}
\begin{verbatim}
### INTERACTING AND COMMUNICATING WITH OTHER GIS ###
 
library(spgrass6)   # GRASS
library(RPyGeo)     # ArcGis (Python)
library(RSAGA)      # SAGA
library(spsextante) # Sextante
\end{verbatim}
\end{framed}

\subsection{Other useful packages}
\begin{verbatim}
## Other useful packages ##
 
library(Metadata)    # automatically collates data from online GIS datasets
(land cover, pop density, etc) for a given set of coordinates
 
#library(GeoXp)    # Interactive exploratory spatial data analysis
  example(columbus)
  histomap(columbus,"CRIME")
 
library(maptools)
# readGPS
 
library(rangeMapper)    # plotting species distributions, richness and traits
 
 
# Species Distribution Modelling
library(dismo)
library(BIOMOD)
library(SDMTools)
 
library(BioCalc)   # computes 19 bioclimatic variables from monthly climatic
values (tmin, tmax, prec)
 
 \end{verbatim}
 
\newpage

\section{Examples}
\begin{verbatim}

### Examples ###
 
### SPATIAL VECTOR DATA (POINTS, POLYGONS, ETC) ###
 
 
# Example dataset: Get "Laurus nobilis" coordinates from GBIF
laurus <- gbif("Laurus", "nobilis")
# get data frame with spatial coordinates (points)
locs <- subset(laurus, select=c("country", "lat", "lon"))
 
# Making it 'spatial'
coordinates(locs) <- c("lon", "lat")    # set spatial coordinates
plot(locs)
 
# Define geographical projection
# to look for the appropriate PROJ.4 description look here:
http://www.spatialreference.org/
 
crs.geo <- CRS("+proj=longlat +ellps=WGS84 +datum=WGS84")    # geographical,
datum WGS84
proj4string(locs) <- crs.geo     # define projection system of our data
summary(locs)
 
 
# Simple plotting
data(wrld_simpl)
summary(wrld_simpl)     # Spatial Polygons Data Frame with country borderlines
plot(locs, pch=20, col="steelblue")
plot(wrld_simpl, add=T)
 
 
### Subsetting
table(locs@data$country)     # see localities by country
 
locs.gr <- subset(locs, locs$country=="GR")   # select only locs in Greece
plot(locs.gr, pch=20, cex=2, col="steelblue")
plot(wrld_simpl, add=T)
summary(locs.gr)
 
locs.gb <- subset(locs, locs$country=="GB")    # locs in UK
plot(locs.gb, pch=20, cex=2, col="steelblue")
plot(wrld_simpl, add=T)
\end{verbatim}
\newpage
\section{Making Maps}
\begin{verbatim}
### MAKING MAPS ###
 
# Plotting onto a Google Map using RGoogleMaps
PlotOnStaticMap(lat = locs.gb$lat, lon = locs.gb$lon, zoom= 10, cex=1.4, pch=
19, col="red", FUN = points, add=F)
 
 
# Downloading map from Google Maps and plotting onto it
map.lim <- qbbox (locs.gb$lat, locs.gb$lon, TYPE="all")
mymap <- GetMap.bbox(map.lim$lonR, map.lim$latR, destfile = "gmap.png",
maptype="satellite")
# see the file in the wd
PlotOnStaticMap(mymap, lat = locs.gb$lat, lon = locs.gb$lon, zoom= NULL,
cex=1.3, pch= 19, col="red", FUN = points, add=F)
 
# using different background
mymap <- GetMap.bbox(map.lim$lonR, map.lim$latR, destfile = "gmap.png",
maptype="hybrid")
PlotOnStaticMap(mymap, lat = locs.gb$lat, lon = locs.gb$lon, zoom= NULL,
cex=1.3, pch= 19, col="red", FUN = points, add=F)
 
# you could also use function gmap in "dismo"
gbmap <- gmap(locs.gb, type="satellite")
locs.gb.merc <- Mercator(locs.gb)    # Google Maps are in Mercator projection.
This function projects the points to that projection to enable mapping
plot(gbmap)
points(locs.gb.merc, pch=20, col="red")
 
 
### Plotting onto a Google Map using googleVis (internet)
points.gb <- as.data.frame(locs.gb)
points.gb$latlon <- paste(points.gb$lat, points.gb$lon, sep=":")
map.gb <- gvisMap(points.gb, locationvar="latlon", tipvar="country",
                  options = list(showTip=T, showLine=F, enableScrollWheel=TRUE,
                           useMapTypeControl=T, width=1400,height=800))
plot(map.gb)
print(map.gb)    # HTML suitable for a web page
 
##########
 
 
# drawing polygons and polylines
mypolygon <- drawPoly()    # click on the map to draw a polygon and press ESC
when finished
summary(mypolygon)    # now you have a spatial polygon!
 
\end{verbatim}
\section{READING AND SAVING DATA}
\begin{framed}
\begin{verbatim}
  
### READING AND SAVING DATA
 
### Exporting KML
writeOGR(locs.gb, dsn="locsgb.kml", layer="locs.gb", driver="KML")
 
### Reading kml
newmap <- readOGR("locsgb.kml", layer="locs.gb")
 
### Saving as a Shapefile
writePointsShape(locs.gb, "locsgb")
 
### Reading (point) shapefiles
gb.shape <- readShapePoints("locsgb.shp")
plot(gb.shape)
 
# readShapePoly   # polygon shapefiles
# readShapeLines  # polylines
# see also shapefile in "raster"
 
 \end{verbatim}
\end{framed}

\newpage
\begin{verbatim} 
 
### PROJECTING ###
 
summary(locs)
# define new projection; look parameters at spatialreference.org
crs.laea <- CRS("+proj=laea +lat_0=52 +lon_0=10 +x_0=4321000 +y_0=3210000
+ellps=GRS80 +units=m +no_defs")
locs.laea <- spTransform(locs, crs.laea)
plot(locs.laea)
 
# Projecting shapefile of countries
country <- readShapePoly("ne_110m_admin_0_countries", IDvar=NULL,
proj4string=crs.geo)     # downloaded from Natural Earth website
plot(country)    # in geographical projection
country.laea <- spTransform(country, crs.laea)  # project
 
# Plotting
plot(locs.laea, pch=20, col="steelblue")
plot(country.laea, add=T)
# define spatial limits for plotting
plot(locs.laea, pch=20, col="steelblue", xlim=c(1800000, 3900000),
ylim=c(1000000, 3000000))
plot(country.laea, add=T)
 
#####################
 
### Overlay
 
ov <- overlay(locs.laea, country.laea)
countr <- country.laea@data$NAME[ov]
summary(countr)
 
 \end{verbatim}
 
\newpage

\subsection{Spline Interpolation}
\begin{verbatim} 

### SPLINE INTERPOLATION
xy <- data.frame(xyFromCell(tmin1.lowres, 1:ncell(tmin1.lowres)))    # get
raster cell coordinates
View(xy)
vals <- getValues(tmin1.lowres)
require(fields)
spline <- Tps(xy, vals)    # thin plate spline
intras <- interpolate(tmin1.c, spline)
intras
plot(intras)
intras <- mask(intras, tmin1.c)
plot(intras)
 
 
# SETTING ALL RASTERS TO THE SAME EXTENT, PROJECTION AND RESOLUTION ALL IN ONE
library(climstats)
?spatial_sync_raster
 
 \end{verbatim}

\subsection{Elevations}
\begin{verbatim} 
### ELEVATIONS: Getting slope, aspect, etc
elevation <- getData('alt', country='ESP')
x <- terrain(elevation, opt=c('slope', 'aspect'), unit='degrees')
plot(x)
 
slope <- terrain(elevation, opt='slope')
aspect <- terrain(elevation, opt='aspect')
hill <- hillShade(slope, aspect, 40, 270)
plot(hill, col=grey(0:100/100), legend=FALSE, main='Spain')
plot(elevation, col=rainbow(25, alpha=0.35), add=TRUE)
 
 
### SAVING AND EXPORTING DATA
 
# writeraster
writeRaster(tmin1.c, filename="tmin1.c.grd")   # can export to many different
file types
writeRaster(tmin.all.c, filename="tmin.all.grd")
 
# exporting to KML (Google Earth)
tmin1.c <- raster(tmin.all.c, 1)
KML(tmin1.c, file="tmin1.kml")
KML(tmin.all.c)     # can export multiple layers
 \end{verbatim}
 
\newpage
\section{References}
\begin{verbatim}
### To learn more ###
 
# Packages help and vignettes, especially
http://cran.r-project.org/web/packages/raster/vignettes/Raster.pdf
http://cran.r-project.org/web/packages/dismo/vignettes/sdm.pdf
http://cran.r-project.org/web/packages/sp/vignettes/sp.pdf
 
# CRAN Task View: Analysis of Spatial Data
http://cran.r-project.org/web/views/Spatial.html
 
# R-SIG-Geo mailing list
https://stat.ethz.ch/mailman/listinfo/R-SIG-Geo
 
# R wiki: tips for spatial data
http://rwiki.sciviews.org/doku.php?id=tips:spatial-data&s=spatial
 
# book
http://www.asdar-book.org/
 
############################################################################

Created by Pretty R at inside-R.org


\end{verbatim}

\end{document}
